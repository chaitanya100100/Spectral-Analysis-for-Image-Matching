\documentclass{beamer}
%
% Choose how your presentation looks.
%
% For more themes, color themes and font themes, see:
% http://deic.uab.es/~iblanes/beamer_gallery/index_by_theme.html
%
\mode<presentation>
{
  \usetheme{default}      % or try Darmstadt, Madrid, Warsaw, ...
  \usecolortheme{default} % or try albatross, beaver, crane, ...
  \usefonttheme{default}  % or try serif, structurebold, ...
  \setbeamertemplate{navigation symbols}{}
  \setbeamertemplate{caption}[numbered]
} 

\usepackage[english]{babel}
\usepackage[utf8x]{inputenc}

\title[Image Graph Spectrum]{Image Graph Spectrum}
\author{Chaitanya Patel}
\institute{CVIT, IIIT-Hyderabad}
\date{June, 2017}

\begin{document}

%-------------------------
% Title Page
%-------------------------
\begin{frame}
  \titlepage
\end{frame}

%-------------------------
% Index
%-------------------------
\begin{frame}{Outline}
  \tableofcontents
\end{frame}


%-------------------------
% Introduction Section
%-------------------------
\section{Introduction to Image Graph}

\begin{frame}{Introduction to Image Graph}

\begin{itemize}
  \item For image segmentation, clustering, etc. \texttt{Graph Spectral Analysis} has been extensively used.
  \item Image is represented as a weighted complete graph. Each pixel represents a vertex.
  \item Edge weights are assigned as per affinity between pixels. Affinity can be defined on the basis of various properties :
  \begin{itemize}
  	\item Intensity difference between two vertices
    \item Gradient difference between two vertices
    \item Difference between some other feature calculated at each pixel e.g. sift
  \end{itemize}

\end{itemize}

%\begin{block}{Examples}
%Some examples of commonly used commands and features are %included, to help you get started.
%\end{block}

\end{frame}



%-------------------------
% Image Graph Section
%-------------------------
\section{Image Graph}

\subsection{Definition of Image Graph}
\begin{frame}{Image Graph}

\begin{itemize}
	\item Image Graph is represented as $G(V, E, W)$
    \item $V$ contains all image pixels as vertices. If there are total $n$ pixels in the image then $|V| = n$
    \item $E$ contains all pairwise relationship between every pair of vertices(pixels) thus making $G$ a complete graph. $|E| = \binom{n}{2}$ for undirected graph
    \item The weight $w_{ij} ≥ 0$ associated with an edge $(v_i , v_j ) \in E$ encodes the affinity between the pixels represented by vertices $v_i$ and $v_j$. We can collect these weights into an $n \times n$ affinity matrix $W = (w_{ij})_{i,j=1,\ldots,n}$
    
\end{itemize}

\end{frame}


\subsection{Function p}
\begin{frame}{Function p}

\begin{itemize}

	\item We want to define a function $p : V \rightarrow \mathbb{R}$ such that it is a continuous function i.e. difference between $p(v_i)$ and $p(v_j)$ inversely follows $w_{ij}$
    \item It is equivalent to say that we want to minimize $$ \lambda = \sum_{i=1}^{n} \sum_{j=1}^{n} w_{ij}(p(v_i) - p(v_j))^2 $$
    \item Let Matrix P be defined as 
    	$\begin{bmatrix}
        	p(v_1)\\
            p(v_2)\\
            \vdots \\
            p(v_{|V|})
		\end{bmatrix}$
\end{itemize}

\end{frame}



%-------------------------
% Laplacian Section
%-------------------------
\section{Laplacian}

\subsection{Incident Matrix}
\begin{frame}{Incident Matrix}

\begin{itemize}

	\item For any directed graph $G(V, E)$, consider $$V = \{v_1, v_2, \ldots\, v_{|V|}\}$$  $$E = \{e_1, e_2, \ldots, e_{|E|}\}$$ 
    \item Incident Matrix $\nabla$ is $|E| \times |V|$ matrix such that if $k^{th}$ edge is from $v_i$ to $v_j$ with weight $w_{ij}$ then
    \begin{itemize}
    	\item $\nabla_{ki} = +w_{ij}$ 
        \item $\nabla_{kj} = -w_{ij}$
        \item $\nabla_{km} = 0, \forall m \neq i, j$
	\end{itemize}
\end{itemize}

\end{frame}


\subsection{Laplacian Matrix}
\begin{frame}{Laplacian Matrix}

\begin{itemize}
	\item $L$ = $\nabla^T\nabla$ is called laplacian of graph
    \item $L$ is $|V| \times |V|$ matrix where 
      $$ L_{ii} = \sum_{j=1}^{|V|} w_{ij} $$
      $$ \underset{i \neq j} {L_{ij}} = -w_{ij} $$
    \item $L = D - W$ where $D$ is degree matrix and $W$ is adjacency matrix
\end{itemize}

\end{frame}


%-------------------------
% Obtaining optimum p Section
%-------------------------
\section{Obtaining optimum p}

\subsection{Laplacian's relation to function p}
\begin{frame}{Laplacian's relation to function p}

\begin{itemize}

	\item We can show that $P^TLP = \frac{\lambda}{2}$
  	\begin{align*}
  	P^TLP &= P^T(D - W)P \\ 
          &= P^TDP - P^TWP
  	\end{align*}
	\item Take $d_{ii}$ $=$ ($i^{th}$ diagonal entry in $D$) and $p_i$ $=$ $p(v_i)$
	\item First term is
    	\begin{align*}
        	P^TDP &= \sum_{i=1}^{|V|}d_{ii} p_{i}^2
        \end{align*}
	\item second term is
    	\begin{align*}
        	P^TWP &= \sum_{i=1}^{|V|} \sum_{j=1}^{|V|} p_{i} p_{j} w_{ij}
    	\end{align*}
        
\end{itemize}

\end{frame}



\begin{frame}{Laplacian's relation to function p (cont.)}

  \begin{align*}
  	P^TLP &= \sum_{i=1}^{|V|}d_{ii} p_{i}^2 - \sum_{i=1}^{|V|} \sum_{j=1}^{|V|} p_{i} p_{j} w_{ij} \\
          &= \sum_{i=1}^{|V|} \sum_{j=1}^{|V|} w_{ij} p_{i}^2 - \sum_{i=1}^{|V|} \sum_{j=1}^{|V|} p_{i} p_{j} w_{ij} \\
          &= \frac{1}{2} ( \sum_{i=1}^{|V|} \sum_{j=1}^{|V|} w_{ij} p_{i}^2 + \sum_{i=1}^{|V|} \sum_{j=1}^{|V|} w_{ij} p_{j}^2 - 2\sum_{i=1}^{|V|} \sum_{j=1}^{|V|} p_{i} p_{j} w_{ij} ) \\
          &= \frac{1}{2} \sum_{i=1}^{|V|} \sum_{j=1}^{|V|} w_{ij} (p_i - p_j)^2 \\
          &= \frac{\lambda}{2}
  \end{align*}

\end{frame}


\subsection{eigenvectors of L as p}

\begin{frame}{Courant-Fischer Formula}

\begin{itemize}
	\item Courant-Fischer Formula for any $n \times n$ symmetric matrix $A$
    \begin{align*}
    	\lambda_1 &= {\min}_{\substack{||x|| = 1}} ( x^TAx ) \\
        \lambda_2 &= {\min}_{\substack{||x|| = 1 \\ x \perp v_1}} ( x^TAx ) \\
         & \vdots \\
        \lambda_{max} &= {\max}_{\substack{||x|| = 1}} ( x^TAx ) \\
    \end{align*} Here $v_1, v_2, v_3, \cdots$ are eigenvectors corresponding to eigenvalues $\lambda_1, \lambda_2, \lambda_3, \cdots$ where $\lambda_1 \leq \lambda_2 \leq \lambda_3 \cdots$
    
    \item We know $P^TLP = \frac{\lambda}{2}$ and $L$ is symmetric. Thus eigenvectors of $L$ corresponding to smallest eigenvalues, represent such possible $P$'s that minimizes $\lambda$

\end{itemize}

\end{frame}


%-------------------------
% Conclusion section
%-------------------------
\section{Conclusion}

\begin{frame}{Conclusion}

\begin{itemize}
	
    \item For any image, a weighted graph is constructed considering each pixel a vertex.
    \item Edge weights are assigned according to affinity of vertices.
    \item Laplacian is obtained from adjacency matrix using formula $L = D - W$
    \item Normalized laplacian can be obtained by formula $L = I - D^{-\frac{1}{2}}WD^{-\frac{1}{2}}$
    \item Eigen Decomposition of $L$ gives $v_1$ as a trivial solution and $v_2, v_3, \cdots$ as desired solutions
    
\end{itemize}

\end{frame}

\end{document}
